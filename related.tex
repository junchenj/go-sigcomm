\tightsection{Related Work}
We briefly highlight two areas of research to place our work in context.

\tightsubsection{QoS of Internet video}
\myparatight{Quality metrics} 
A key aspect of video delivery is to optimize user-perceived  quality of experience.  There is evidence that
users are sensitive to buffering ratio (e.g., \cite{sigcomm11}) and to frequent changes in bitrate (e.g.,~\cite{user-adaptive,videoqoe}). More recent studies show a causal relationship between quality and user engagement~\cite{akamai-imc12}. The design of a good QoE metric (e.g., \cite{qscore}) is still an active area of research. As our understanding of video QoE matures, we can extend GO to be QoE-aware (as suggested in~\cite{sekar2012quest}).

\myparatight{Client-side adaptation} This includes work on identifying problems in existing client-adaptation algorithms (e.g.,~\cite{mmsys2011cisco}), understanding the interaction between client logic and TCP (e.g.,~\cite{usenix12_ghobadi,nossdav12_esteban}), and designing better bitrate adaptation algorithms to improve quality (e.g.,~\cite{nossdav12_akhshabi,festive}). GO can be considered an enhancement to client-side adaptation that makes decisions based on the information collected from many clients.

\myparatight{CDN and server selection} These efforts have demonstrated inefficiencies in CDN and server selection strategies~\cite{youtubecdn,youtube-infra}, and studied various techniques to improve video quality, including cross-CDN optimization (e.g.,~\cite{sigcomm12cdnmulti}), CDN federation (e.g.,~\cite{peterson2013framework}) and the use of ISP-CDN (e.g.,~\cite{yu2012tradeoffs,frank2013pushing}).  While GO currently focuses on quality improvement via better initial CDN and bitrate selection, we expect GO to help these techniques make more accurate decisions. \cite{sigcomm12} is very relevant as it also envisions a global control plane for video, but it lacks a concrete design and performance analysis. 

\myparatight{Other video measurement}  There is a large literature on measuring content popularity and access patterns (e.g.,~\cite{youtube-imc07,plissonneau2012longitudinal}) and flash crowds (e.g.,~\cite{beijing-imc09}). Using a dataset that provides a similar panoramic view,~\cite{jiang2013shedding} studies the structure of video quality problems in more dimensions (e.g., content provider and video type). GO's attribute-based prediction algorithm extends a similar structural analysis to quality prediction.

\myparatight{Quality prediction/diagnosis} GO's prediction algorithm uses dynamic aggregation. Though we largely borrow the idea from~\cite{george2008value}, aggregation is a widely-used technique in statistical prediction. \cite{stemm2000network} provides a good review of aggregation techniques. In the domain of video, previous work has focused on quality diagnosis using the idea of hierarchical heavy hitters (e.g.,~\cite{hhh,iptv-sigcomm09}).

\tightsubsection{Global optimization}

\myparatight{Control plane and centralized decisions} There is a long line of work in which the control plane is separated from the data plane (e.g.,~\cite{rcp,onix,yan2007tesseract,openflow}), focusing on physical networks with the objectives of enforcing packet forwarding behaviors and policy compliance in routing. In contrast, GO optimizes at the application level by controlling client-side devices, and its objectives are quality metrics.

\myparatight{Controller scalability and fault-tolerance} By making control decisions at a centralized point, SDN and similar centralized systems create a single point-of-failure at the controller and previous research has proposed many related techniques to build a scalable and fault-tolerant controller/backend (e.g., ~\cite{tootoonchian2012controller,yan2007tesseract}). The implementation of GO backend leverages many existing techniques and software artifacts to make it scalable, efficient and resilient, including~\cite{spark,hadoop,zaharia2012resilient,kreps2011kafka}

\myparatight{Information sharing and data-driven decision-making} Information sharing has been used for decades to make more informed decisions in networking research. These works (e.g.,~\cite{stemm2000network} and more recently,~\cite{choffnes2010crowdsourcing}) explore the feasibility of a wide area measurement infrastructure to monitor network-level performance and inform application-level decisions. Sharing across multiple applications within a host is also studied (e.g., \cite{balakrishnan1999integrated}). They also suggest the ineffectiveness of static models and argue for data-driven approaches (e.g.,~\cite{Winstein:2013:TEM:2486001.2486020, sivaraman2013no}) to make parameters dynamically configured.
These works try to optimize quality like GO, but they require dedicated centralized sharing or distribute sharing infrastructure, while we rely only on instrumentation within video players to measure wide-area performance, leveraging the dominance of video traffic in today's Internet. Moreover, GO deals with more attributes and requires a more complex model than these works whose network models do not include application-specific attributes, e.g., content provider or streaming protocols. 
