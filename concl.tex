\section{Conclusion}

In this paper, we presented the case for and an early instantiation of a global control platform to manage QoS-sensitive Internet applications 
such as Internet video streaming. To handle the ever-increasing choice space (e.g., on CDN, video encoding, video bitrate, server, path or route, etc), 
we take a predictive and data-driven approach that can choose better initial configurations and respond to Internet events more quickly than the reactive
approach that most Internet applications rely on today. In video delivery, this can be achieved by collecting information from all elements in the 
eco-system (video sessions, servers, routers, etc) and predict the quality metrics of a new session given its attributes.
Building such systems in practice, however, is challenging due to prediction errors.
We presented the design of GO, a global video control platform that addresses these challenges and provides near-optimal quality metrics that GO optimizes. 
We also share our early experience with deploying GO on one video site. 
While the potential of GO is still constrained by deployment limitations and business policies, 
it is nevertheless a very promising first step towards a global control platform.

