\begin{abstract}
With more and more interactive and video applications being deployed
  over the Internet, the demand for Quality of Service is continuously
  increasing. Unfortunately, most of today's Internet applications and
  protocols (e.g., TCP and various media/video protocols) are designed
  to \emph{react} to congestion and/or changes in the resource
  availability. These reactive approaches lead to suboptimal decisions
  as they may take too long to converge, or react long after the
  quality has been already impacted. 

  In this paper we make two arguments. First, we argue that 
  being able to accurately predict the outcomes of the available choices 
  (e.g., would a streaming video client be able to sustain a particular bitrate?) 
  can greatly help to achieve better QoS. Second, we argue that to
  accurately predict the outcome of a given choice, we need to
  leverage the information available from other sessions, streams or
  connections.

  We further propose a global control architecture that
  continuously collects performance information of ongoing sessions
  and uses this information to accurately predict the performance
  outcome of sessions for a given choice. In particular, we present
  an instantiation of such a global architecture for video streaming,
  called Video Global Optimization (GO). GO instruments each client
  streaming video to send quality-related information to a backend,
  which processes the quality information in real-time and provides
  hints to clients about the best bitrate or CDN to start with.
  We implemented GO and deployed it on one video site, and demonstrated 
  that by making initial bitrate and CDN selections, GO can simultaneously
  increase the average bitrate and reduce the amount of time spent
  buffering.  Improvement is 10-20\% on average, and up to 30\% for certain
  classes of clients. While our first GO deployment cannot achieve its potential
  due to business policies and incremental deployment constraints, 
  we demonstrated that it is a promising first step towards a full control platform.

%Despite the increasing demand for better Quality of Service, today's Internet applications and protocols (e.g., TCP and various media/video protocols) are designed to react to congestion and/or changes in the resource availability. However, these reactive protocols typically makes suboptimal decision (e.g., static initial configurations) and takes relatively long time to converge by reaction both negatively impacting quality and user experience. This paper makes two arguements. First, we argue that to achieve of better quality, one needs to accurately predict the outcome of making a particular choice (e.g., whould a stream be able to sustain a particular bitrate?). Second, we argue that to accurately predict the outcome of a given choice, we need to leverage the information available from other streams or connections. We further propose a global control architecture that continuously collects performance information of ongoing sessions and uses this information to accurately predict the performance outcome of a sessions for a given choice.

%In this paper, we present an instantiation of such global architecture for video streaming, called Video Global Optimization (GO). GO instruments each client streaming video to send quality related information to a backend which processes the quality information in real-time and provides hints to clients about the best bitrate or CDN to start with or switch to. We present the system architecture and discuss in detail the challenges and solutions for making accurate predictions based on aggregate quality information received by the backend. Using standard quality metrics, we demonstrate that by making initial bitrate and CDN selection, GO can improve average bitrate and reduce buffering time simultaneously with a magnitude of up to \fillme in some ASN and CDN.

\end{abstract}
