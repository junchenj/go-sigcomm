\tightsection{Prediction algorithms}
\label{prediction}
The tradeoff between estimation error and bias via aggregation naturally leads us to consider a class of algorithms that compute average quality outcomes for groups of sessions under different attribute sets, and then {\it dynamically} choose the attribute set that seems to work well for a given session.  Since it is difficult to find the best attribute set, it is useful to hedge our bets by taking a weighted combination of averages.  George et al \cite{george2008value} consider this problem and propose the following algorithm for choosing the weights.  For a session under prediction having index $i$, let $a_i$ be the tuple of all attributes observed for that session and let $q_i$ be its quality outcome.  For each attribute set $g$ in our collection of attribute sets $G$, define a function $v_g$ that picks out those attributes:
\begin{equation*}
  v_g(\cdot): a \mapsto \text{[the subtuple of $a$ containing attribute values} \\
  \text{for the attributes in $g$]} .
\end{equation*}
We first compute the single-group average quality outcome for each $g$:
\begin{align*}
  \bar{q}_{g}(a_i) &= \frac{1}{N_{g}(a_i)} \sum_{j \in N_{g}(a_i)} q_j, \text{ where} \\
  N_{g}(a_i) &= |\{j: v_g(a_j) = v_g(a_i), j < N\}|
\end{align*}
Then, considering $\bar{q}_{g}(a_i)$ as an estimate of $q_i$, estimate its mean squared prediction error $e_{g}(a_i)$.  This estimation step is not simple, so we will briefly postpone its description.  Having computed $\bar{q}_{g}(a_i)$ and $e_{g}(a_i)$ for each attribute set, our final quality prediction, which we denote $\bar{q}(a_i)$, is the weighted average of $\bar{q}_{g}(a_i)$, where the weights are equal to the inverse of $e_{g}(a_i)$, normalized to sum to $1$:
\begin{align*}
  \bar{q}(a_i) = (\sum_{g \in G} e_{g}(a_i)^{-1})^{-1} \sum_{g \in G} e_{g}(a_i)^{-1} \bar{q}_{g}(a_i) .
\end{align*}

$e_{g}(a_i)$ is estimated by separately estimating (squared) bias and estimation error and summing them, plugging them into equation \eqref{eqn:biasvariance}.  Bias is heuristically taken to be the difference between the finest-granularity estimate of $q_i$ (i.e. $\bar{q}_{g'}(a_i)$, where $g'$ is the finest-grained ACS) and $\bar{q}_{g}(a_i)$.  Variance is estimated by a standard formula:
\begin{align*}
  \frac{1}{N_{g}(a_i)} \sum_{j \in N_{g}(a_i)} (q_j - \bar{q}_{g}(a_i))^2
\end{align*}

George et al call the resulting prediction algorithm WIMSE, for Weighted Inverse Mean Squared Error.  Inverse-mean-squared-error weighting has the following desirable property: If each $\bar{q}_{g}(a_i)$ is statistically independent, then $\bar{q}$ is an optimal estimator in the sense that it has minimal mean squared error among all functions of \cite{}.  However, in our setting (and in the original motivating setting) the $\bar{q}_{g}(a_i)$ values are based on overlapping data and are therefore not independent.  Furthermore, our estimates of both the bias and variance components of $e_{g}(a_i)$ are imprecise. While this doesn't allow us to provably guarantee WIMSE's optimality, George et al have found that WIMSE can nevertheless work well in practice.  To optimize the algorithm for out setting we made a few changes, which we now present. 

\begin{figure*}[t!]
\centering
\subfigure[Exhaustive vs. greedy (GO) search]
{
        \includegraphics[width=0.3\textwidth]{figures/prediction-comparisons/example-exhaustive-metric1.pdf}
}
\subfigure[GO vs. oracle]
{
        \includegraphics[width=0.3\textwidth]{figures/prediction-comparisons/example-oracle-metric1.pdf}
}
\subfigure[Granular of ACS selection]
{
        \includegraphics[width=0.3\textwidth]{figures/prediction-comparisons/example-granular-metric1.pdf}
}
\tightcaption{Comparison between GO ACS selection with other strawmans (considerd quality metric is average bitrate).}
\label{fig:acs-comparison}
\end{figure*}

\tightsubsection{ACS selection}
WIMSE may not perform best when given the power set of possible ACs (in total, $2^k$ in case of $k$ attributes) because the independence assumption is dramatically violated.  For example, if two ACs produce similar groups, then both will receive similar weights; including both ACs doubles the weight of the corresponding groups for no principled reason.
 Another reason to choose ACs carefully is that many of the computational costs of WIMSE scale with the number of groups; we discuss this further in \Section~\ref{sec:scalability} \jc{make sure we cover it}.

Therefore, we pick an appropriate AC Set (which we call ACS) that produces the minimal empirical prediction error. 
We take a data-driven approach to ACS selection. We separate part of historical sessions as test set. Then, the best ACS is the one that gives the minimal overall prediction error over the test set using WIMSE with this ACS.

\myparatight{Greedy search vs. exhaustive search} One problem with this approach is that even in offline analysis, the potential space to search for the best ACS is too huge (with $k$ attributes, there will be $2^k$ ACs, and in total $2^{2^k}$ ACSes). Instead, we first examine a greedy algorithm which finds a good ACS in $O(2^k)$ steps, and compare it against exhaustive search result for $k=3$ attributes on which exhaustive search is possible. Figure~\ref{fig:acs-comparison}(a) presents the CDF of prediction accuracy w.r.t average bitrate of the ACS that obtained by exhaustive search and greedy search. It shows that the ACS they choose, though different, can offer similar quality of prediction.


\myparasum{Per-finest group selection vs. one global ACS} One strawman for our algorithm to compare with is one where the selected ACS can be different for different sessions. Especially, if two sessions differ in at least one attribute, they can use different ACS. 
We compare the overall prediction error in Figure~\ref{fig:acs-comparison}(c) with the same setup as last paragraph, and they perform similarly and the global ACS performs even slightly better, possibly because per-finest-group ACS has side-effect of overfitting.

\myparasum{Oracle approach vs. historical based} Now, to reveal the optimality of GO, we compare its overall prediction error with the predictability i.e., minimal prediction error (see \Section~\ref{sec:predictability}) as well as the optimal AC. Recall that the optimal AC reflects the best possible prediction using the best AC of historical data based on the quality informaton of sessions under prediction, while predictability represents the best possible prediction in theory with no constraint on the way we use historical data. Figure~\ref{fig:acs-comparison}(b) shows that GO performs very similarly to optimal AC and such accuracy is also close to optimal (predictability) in most cases. \jc{predictability line needs to be added}

\tightsubsection{Pseudocount priors}
That algorithm assumes that bias and variance can be computed exactly, while in practice variance cannot be estimated accurately when groups are very small. For instance, for a group sessions, it is very likely to have all sessions with no start failure and prediction based on this information will be greatly biased.
To alleviate this, we use a simple idea from Bayesian statistics: We incorporate a prior distribution on quality outcomes within each group.  This amounts to adding a few fake ``pseudocount'' observations to each group, in a very similar way of Bayesian count~\cite{}.  Figure~\ref{fig:sudocount} shows the improvement on prediction error of start failure using pseudocount prior.


\begin{figure}[h!]
\centering
 \includegraphics[width=0.4\textwidth] {figures/prediction-comparisons/example-pcount-metric3.pdf}
\tightcaption{GO with pseudocount vs. without pseudocount (considerd quality metric is average bitrate).}
\label{fig:sudocount}
\end{figure}


\tightsubsection{Mini-evaluation of prediction accuracy}
Given an ACS selection algorithm, we're going to use all attributes we have and test GO prediction accuracy against two simple baselines.
\begin{itemize}
	\item GO with greedy ACS selection and pseudocount priors
	\item Predict using average of each CDN and bitrate combination: picking the decision that is globally best (i.e., no spatial partition)
	\item Predict using average of the finest group that has any data.
\end{itemize}
Figure~\ref{fig:compare-to-naive}


All these results indicate that GO performs better than both baseline. This confirms the observation in \Section~\ref{subsec:aggregation} that the best level of aggregation is often dynamic and in between the coarsest and finest aggregation.

\begin{figure*}[t!]
\centering
\subfigure[Buffering ratio]
{
        \includegraphics[width=0.22\textwidth]{figures/prediction-comparisons/example-naive-metric0.pdf}
}
\subfigure[Average bitrate]
{
        \includegraphics[width=0.22\textwidth]{figures/prediction-comparisons/example-naive-metric1.pdf}
}
\subfigure[Join time]
{
        \includegraphics[width=0.22\textwidth]{figures/prediction-comparisons/example-naive-metric2.pdf}
}
\subfigure[Start failure]
{
        \includegraphics[width=0.22\textwidth]{figures/prediction-comparisons/example-naive-metric3.pdf}
}
\tightcaption{Comparison between GO and two baseline approaches where the selection and use of ACS is statically used -- (1) no attributes considered except (CDN, bitrate), (2) all attributes considered.}
\label{fig:compare-to-naive}
\end{figure*}



\tightsubsection{Interactions between decisions\jc{Move to discussion}}
The reader may be bothered by a simplifying assumption implicit in our characterization of the causes of prediction error.  If we allocated all traffic to a single CDN, its performance might degrade, but our session-wise prediction does not capture that.  We might instead want to know the following: Given a set of decisions about sessions (say, all the sessions we observe in a 1-hour interval), what is the predicted performance for that set of sessions?  We do not wish to say that such a question is impossible to answer, but rather point out some difficulties in answering it, and some reasons why it is less critical to answer it than it may appear.

Joint prediction is statistically difficult because existing statistical prediction algorithms typically assume the performance of training examples are independent and experience identical randomness (i.e. they are IID).  We observe very few IID instances of whole sets of sessions; in our example, we observe only one per 1-hour interval.  It is possible to model explicitly the dependence of each session’s performance on the set of joint decisions, but this requires modeling choices that we may not make well, and such models are typically computationally expensive to learn. \fillme

If conditions change slowly enough, the independence assumption is not so bad.
The rate of change of CDN allocations is naturally limited for our problem by the rate of session arrivals, since we choose CDNs only at the beginning of each session.  Spikes in the rate of new sessions are not typically high enough to necessitate special handling.  \henry{Need some data and experiments for this.  Davis or Florin have done some of the experiments, I think.}


\tightsubsection{Alternative approaches\jc{Move to discussion}}
The reader should not leave with the impression that the algorithm described above is the only possible one for predicting video quality, or even the best.  The scope of this paper is merely to establish a reasonable approach and show that it results in improvements.  Other possible approaches might include:
\begin{packedenumerate}
  \item \emph{Linear regression:} After encoding categorical attributes as binary features, simple linear regression can be applied to predict quality outcomes.  Temporal attributes can be passed through nonlinear functions to achieve reasonable time-series prediction.  Interaction terms (e.g. the indicator for a session coming from ASN $100$ multiplied by the indicator for the session having Object ``foo'') can simulate attribute combinations, at the cost of a combinatorial explosion in the size of the learned model.  However, recent developments in optimization for $\ell_1$-regularized linear regression allow models to be learned quickly online while providing the guarantee that the learned model is \emph{sparse}, i.e. that only a few important features are selected for inclusion in the model and the rest can be safely dropped.  See \cite{duchi2010composite} for one example of work that could enable this technique.  One downside is that linear models are harder to interpret than a model based on averaging group averages.
  \item \emph{Hierarchical Bayesian modeling:} In this approach, groups are placed in the natural tree, and each is associated with a probability distribution over quality outcomes, such as a Gaussian distribution.  Each group inherits information from the distribution of its parent group in the form of a prior.  Such models potentially deal very naturally with data sparsity and with dependence among sessions \cite{gelman2003bayesian}, but learning them from data is often computationally intractable.
\end{packedenumerate}